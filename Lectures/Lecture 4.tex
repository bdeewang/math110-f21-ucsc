\vspace{1em}

{\bf\large Distribution of prime numbers I. \emph{Larger scale}}
\vspace*{0.5em}
\begin{theorem}[Euclid]
There are infinitely many primes.
\end{theorem}
\begin{proof}
Towards a contradiction, assume there are finitely many primes
\[p_1 = 2,\, p_2 = 3,\ldots,\,p_N\]
Consider $M = p_1p_2\cdots p_N + 1$, since every positive integer has a prime factor, $p_i \mid M$ for some $1\leq i \leq N$.\\[0.5em]
But also note that $p_i \mid p_1p_2\cdots p_N$, and therefore $p_i$ divides $M - p_1p_2\cdots p_N = 1$, giving us a contradiction. Hence, there are infinitely many primes.
\end{proof}

\vspace*{2.5em}

More quantitatively,
\begin{definition}
For any real number $x>0$,
\[\pi(x) \coloneqq \text{number of primes} \leq x\]
$\pi$ is called the \emph{prime counting function}.
\vspace*{1em}\\
\emph{e.g.}\quad $\pi(1.5) = 0$,\ $\pi(23) = \card\set{2,3,5,7,11,13,17,19,23} = 9$.
\end{definition}

\vspace*{2em}

So, Euclid's theorem says that as $\pi(x) \to \infty$ as $x \to \infty$. Computing $\pi(x)$ is hard, one can do it for $x \sim 10^{13}$ but not for, say, $x \sim 10^{10^{10}}$.\\[1em]
\begin{itemize}[leftmargin=4.8em]
\item[\emph{\textbf{Question.}}] Do we have an asymptotic formula for $\pi(x)$?\\[0.5em]
That is, can we find a simpler function $f$ such that 
\[\lim_{x\to \infty}\frac{\pi(x)}{f(x)} = 1\]
If so, can we bound the "error" $|\pi(x) - f(x)|$ in terms of $x$?
\end{itemize}

\vspace*{1em}

\begin{theorem}[Prime Number Theorem, (1896) Hadamard, de la Vall\'ee Poussin]
\[\lim_{x \to \infty}\frac{\pi(x)}{\mathrm{Li}(x)} = 1,\]
where $\displaystyle \mathrm{Li}(x) = \int_2^x \dfrac{dt}{\log t} \sim \dfrac{x}{\log x}$
\end{theorem}

%\vspace*{1em}

\begin{conjecture}[a consequence of the Riemann Hypothesis]
For all $x \geq 2657$,
\[|\pi(x) - \mathrm{li}(x)| < \frac{1}{8\pi}\,\sqrt{x}\log x,\]
where $\displaystyle \mathrm{li}(x) = \int_0^x \dfrac{dt}{\log t} = \mathrm{Li}(x) - \ln 2$.
\end{conjecture}

\vspace*{3em}

{\bf\large Distribution of prime numbers I. \emph{Smaller scale}}
\vspace*{0.5em}
\begin{align*}
\underbracket{2,\ 3}_1,\ 5,\ 7,\ 11,\ 13,\ &\ldots,\ 263,\ 269,\ 271,\ 277,\ \ldots,\\[0.5em]
& 877,\ 881,\ 883,\ 887,\ 907,\ \ldots
\end{align*}
Since all primes, bigger than $2$, are odd, so the gap between them is always even; except between $2$ and $3$.

\vspace*{2em}
 
\begin{definition}
\emph{Twin primes} are a pair of primes $(p,q)$ such that $|p-q| = 2$.
\end{definition}

\vspace*{1em}

\begin{conjecture}
There are infinitely many twin primes.
\end{conjecture}

\vspace*{1em}

\begin{theorem}
There are infinitely many pairs $(p,q)$ of primes such that 
\begin{align*}
\textnormal{($\sim 2013$, Y. Zhang)}\quad & |q-p| < 7\cdot 10^7\\[0.5em]
\textnormal{($\sim 2014$, Polymath8)}\quad & |q-p| < 246
\end{align*}
\end{theorem}

\vspace*{2em}

{\bf\large Sum of Divisor functions.}
\vspace*{0.5em}
\begin{definition}
A function $f:\zz_+ \to \cc$ is called \emph{multiplicative} if $f(ab) = f(a)f(b)$, for coprime positive integers $a,b$.
\vspace{0.5em}\\
\begin{proof}[e.g.]\renewcommand{\qedsymbol}{}
\begin{itemize}
\item[(1)] Fix $k \in \rr$, define $f_k(n) = n^k$. Then $f_k(ab) = (ab)^k = a^kb^k = f_k(a)f_k(b)$.
\end{itemize}
\begin{itemize}[leftmargin=4.4em]
\item[(2)] Recall $\sigma_0:\zz_+ \to \cc$, where $\sigma_0(n) = \text{number of positive divisors of $n$}$. Corollary \ref{multsigma0} tells us that $\sigma_0$ is multiplicative.\\[0.5em]
Note that the coprime assumption is essential; consider, for instance, $a = 2$ and $b = 4$, then $\sigma_0(2\cdot 4) = \sigma_0(8) = 4$. But $\sigma_0(2) = 2$ and $\sigma_0(4) = 3$, therefore $\sigma_0(2)\sigma_0(4) = 6 \neq 4 = \sigma_0(2\cdot 4)$.
\end{itemize}
\begin{itemize}[leftmargin=6em]
\item[\emph{non-example.}] Consider the function $f: \zz_+ \to \cc$ given as $f(n) = 2n+1$. Then note for $a = 2$ and $b = 3$, we have
\[f(2\cdot 3) = f(6) = 11\]
\[f(2) = 3;\quad f(3) = 5\]
Of course, $f(2\cdot 3) = 11 \neq 15 = f(2)f(3)$.
\end{itemize}
\end{proof} 
\end{definition}

%\vspace*{0.2em}

\begin{definition}
Let $k \in \rr$, define 
\[\sigma_k:\zz_+ \to \cc\]
as $\sigma_k(n) =$ sum of $k$-power of positive divisors of $n$, that is
\[\sigma_k(n) = \sum_{d \in \mathscr{D}(n)}d^k\]
\emph{Special Case.} When $k = 0$, then $\sigma_0(n) = \sum_{\,d\in \mathscr{D}(n)}d^0 = \card\,\mathscr{D}(n) = \text{number of positive divisors of $n$}$.\\[1em]
\emph{e.g.}\quad $k = 2,\ \sigma_2(6) = 1^2 + 2^2 + 3^2 + 6^2 = 50$.
\end{definition}

\vspace*{1.5em}

\begin{theorem}
For any $k \in \rr,\ \sigma_k$ is multiplicative.
\end{theorem}
\begin{proof}
Let $a,b$ be positive integers such that $\gcd(a,b) = 1$, then
\begin{align*}
\sigma_k(ab) = \sum_{w\in \mathscr{D}(ab)}w^k &= \sum_{(u,v)\in \mathscr{D}(a) \times \mathscr{D}(b)}(uv)^k ,\quad \text{by Theorem \ref{bijection}}\\[0.5em]
&= \sum_{u \in \mathscr{D}(a)}\sum_{v \in \mathscr{D}(a)}u^kv^k\\[0.5em]
&= \sum_{u \in \mathscr{D}(a)}u^k\left(\sum_{v \in \mathscr{D}(a)} v^k\right)\\[1em]
&= \sum_{u \in \mathscr{D}(a)}u^k\sigma_k(b)\\[0.5em]
&= \sigma_k(b)\left(\sum_{u \in \mathscr{D}(a)}u^k\right) = \sigma_k(b)\sigma_k(a)
\end{align*}
Therefore $\sigma_k(ab) = \sigma_k(a)\sigma_k(b)$, whenever $\gcd(a,b) = 1$.
\end{proof}

\vspace*{1em}

We're now ready to introduce a strategy to compute $\sigma_k(n)$ for any given $k$ and $n$, but first we review a proposition that we need.

\vspace*{0.5em}

\begin{proposition}\label{geomsum}
Let $x\neq 1$ be a real number and $e$ be a non-negative integer. Then 
\[1 + x + x^2 + \cdots + x^e = \frac{x^{e+1}-1}{x-1}\]
\end{proposition}
\begin{proof}
Let $S = 1 + x + x^2 + \cdots + x^{e-1} + x^e$, then
\begin{align*}
xS &= x + x^2 + \cdots + x^e + x^{e+1}\\[0.5em]
&= x + x^2 + \cdots + x^e + x^{e+1} + 1 - 1\\[0.5em]
&= (1 + x + x^2 + \cdots + x^e) + x^{e+1} - 1\\[0.5em]
&= S + x^{e+1} - 1
\end{align*}
Therefore $(x-1)S = xS- S = x^{e+1} - 1$. Since $x\neq 1$, hence $\displaystyle S = \frac{x^{e+1}-1}{x-1}$.
\end{proof}

\vspace*{1em}

\begin{corollary}\label{sigprimeform}
If $p$ is a prime, $e \geq 0$ an integer and $k \neq 0$, then
\[\sigma_k(p^e) = \frac{(p^{e+1})^k - 1}{p^k - 1}\]
\end{corollary}
\begin{proof}
Note that for any $d \in \mathscr{D}(p^e)$, we necessarily have $d = p^f,\ 0 \leq f \leq e$ (in particular, $\sigma_0(p^e) = e + 1$). Therefore
\begin{align*}
\sigma_k(p^e) &= (p^0)^k + (p^1)^k + (p^2)^k + \cdots + (p^{e-1})^k + (p^e)^k\\[0.5em]
&= 1 + p^k + (p^k)^2 + \cdots + (p^k)^e\\[0.5em]
&= \frac{(p^k)^{e+1} - 1}{p^k - 1},\quad \text{taking $x = p^e$ in Proposition \ref{geomsum}}
\end{align*}
since $k \neq 0$.
\end{proof}

\vspace*{1em}

\begin{example}
Compute $\sigma_3(12)$.
\end{example}
\begin{proof}[Answer]
First note that $12 = 2^2\cdot 3$. Since $\gcd(2^2,3) = 1$, therefore 
\begin{align*}
\sigma_3(12) &= \sigma_3(2^2)\sigma_3(3)\\[0.5em]
&= \frac{(2^{2+1})^3 - 1}{2^3 - 1}\cdot\frac{(3^{1+1})^3 - 1}{3^3 - 1} = 73\cdot 28.
\end{align*}
Hence $\sigma_3(12) = 2044$.
\end{proof}

\vspace*{1em}

So our general strategy to compute $\sigma_k(n)$, when $k>0$, is as follows:
\begin{itemize}
\item Consider the prime factorisation of $n$, say $n = p_1^{e_1}\cdots p_r^{e_r}$
\item Since $\gcd(p_i,p_j) = 1$, therefore $\gcd(p_i^{e_i},p_j^{e_j}) = 1$ for $i \neq j$ (see Problem \ref{gcd a}).
\item Hence $\sigma_k(n) = \sigma_k(p_1^{e_1})\cdots\sigma_k(p_r^{e_r})$.
\item Apply Corollary \ref{sigprimeform} to $\sigma_k(p_i^{e_i})$.
\end{itemize}

%\vspace*{1em}

One reason why we care about $\sigma_k$

\vspace*{0.5em}

\begin{definition}
Let $n$ be a positive integer
\begin{itemize}
\item[(1)] Say $n$ is \emph{perfect} if the sum of proper divisors of $n$, that is divisors strictly less than $n$, equals $n$. Equivalently, if $\sigma_1(n) = 2n$.
\item[(2)] Say $n$ is \emph{deficient} if $\sigma_1(n) < 2n$.
\item[(3)] Say $n$ is \emph{abundant} if $\sigma_1(n) > 2n$.
\end{itemize}
\vspace*{0.5em}
\begin{proof}[e.g.]\renewcommand{\qedsymbol}{}
\begin{itemize}
\item[(1)] Let $n = p$ be a prime, then $\sigma_1(p) = 1 + p < 2p$. Therefore primes are deficient.
\end{itemize}
\begin{itemize}[leftmargin=4.4em]
\item[(2)] Let $n = 6$, then $\sigma_1(6) = 1 + 2 + 3 + 6 = 6 + 6$. Therefore $6$ is perfect.
\item[(3)] Let $n = 12$, then $\sigma_1(12) = 1 + 2 + 3 + 4 + 6 + 12 > 24$. Therefore $24$ is abundant.
\end{itemize}
\end{proof}
\end{definition}

\vspace*{-1em}

\emph{Known perfect numbers:} $6,\,28,\,496,\,8128,\,33550336,\ldots$ (about fifty).

\vspace*{1em}

\begin{proof}[Open Questions]\renewcommand{\qedsymbol}{}
\hfill
\begin{itemize}
\item are there infinitely many perfect numbers?
\item are there any \emph{odd} perfect numbers?
\end{itemize}
\end{proof}

%\vspace*{1em}

\begin{example}[in-class]
\hfill
\begin{itemize}
\item[(i)] Express $\sigma_4(27)$ as $\dfrac{3^a - 1}{3^b - 1}$ for some positive integers $a,b$.
\item[(ii)] Suppose we're given a multiplicative function $f$, and we've been told 
\[f(2) = 4,\quad f(3) = 11,\quad f(4) = 3,\quad f(6) = 33,\quad f(8) = 5\]
Do we know enough to compute $f(24)$? If yes, compute it. If not, why not?
\end{itemize}
\end{example}

\vspace*{2em}

\emph{\textbf{Question}}. \textit{For which $n \in \zz_+$, is $2^n - 1$ prime?}

\vspace*{1em}

\begin{proposition}\label{mersprime}
If a positive integer $n$ is such that $2^n - 1$ is prime, then $n$ is prime.
\end{proposition}
\begin{proof}
Suppose, for sake of contradiction, $n$ is not prime. Then $n = ab$, for some $1<a,b<n$, which gives us
\begin{align*}
2^n - 1 &= 2^{ab}-1\\[0.5em]
&= (2^a)^b - 1^b\\[0.5em]
&= (2^a - 1)((2^a)^{b-1}+(2^a)^{b-2}+\cdots+2^a+1)
\end{align*}
These are non-unit proper divisors of $2^{ab} - 1$, since $1<a,b<n$. Therefore this number is composite, giving us a contradiction. Hence $n$ is prime.
\end{proof}

%\vspace*{1em}

\emph{\textbf{The converse is not true}}, that is, if $n$ is prime then $2^n - 1$ is not necessarily prime.
\vspace*{0.5em}\\
\emph{e.g.}\quad $2^{11} - 1 = 2047 = 23\cdot 89$.

\vspace*{0.5in}

\subsection{Problems}
\vspace{0.1in}

\begin{problem}\label{problem 4.1}
For this problem, you may want to review one-variable Calculus.
\begin{itemize}
\item[(a)] Recall the definition 
\[\mathrm{Li}(x) = \int_2^x \frac{dt}{\log t}\quad \text{for } x > 2\]
What is $\dfrac{d}{dx}\mathrm{Li}(x)$?
\item[(b)] Prove that \[\lim_{x\to \infty}\frac{\mathrm{Li}(x)}{x/\log x} = 1.\]
\end{itemize}
\end{problem}

\vspace*{0.1in}

\begin{problem}\label{problem 4.2}
Compute $\sigma_1(N)$ for the three numbers $N = 28,\ N = 111$, and $N = 240$. Classify $N$ as abundant, deficient or perfect.
\end{problem}

%\newpage
\vspace*{0.1in}

\begin{problem}[Mersenne, 1644]\label{problem 4.3}
Describe all circumstances under which $\sigma_1(n)$ is odd.
\end{problem}

\vspace*{0.1in}

\begin{problem}\label{problem 4.4}
Find a short formula for $\sigma_0(p) + \sigma_1(p) + \cdots + \sigma_k(p)$ whenever $p$ is a prime number.
\end{problem}

\vspace*{0.1in}

\begin{problem}\label{problem 4.5}
Prove that for every positive integer $n$, we have
\[\sigma_k(n) = \sigma_{-k}(n)n^k.\]
Conclude that $n$ is perfect if and only if $\sigma_{-1}(n) = 2$.
\end{problem}

\vspace*{0.1in}

\begin{problem}\label{problem 4.6}
Prove that if $n$ is a perfect square, then $n$ is not a perfect number.
\end{problem}

\vspace*{0.1in}

\begin{problem}\label{problem 4.7}
We say that a positive integer $n$ is \emph{square-free} if $n$ is not divisible by $p^2$ for any prime number $p$. Necessarily, $n$ is then a distinct product of primes. (E.g. $15$ and $37$ are square-free, but $24$ and $49$ are not.)\\[0.5em]
Consider the function $\mu$ (named after A.F. M{\"o}bius, popularly known for his strip) defined on positive integers $n$ as follows:
\[\mu(n) = \begin{cases}1 & \text{if $n = 1$}\\[0.5em] 0 & \text{if $n$ is not square-free}\\[0.5em] (-1)^t & \text{if $n$ is square-free and has exactly $t$ prime divisors.} \end{cases}\]
\begin{itemize}
\item[(a)] Compute $\mu(n)$ for $n = 1,\ldots,15$.
\item[(b)] Prove that $\mu$ is multiplicative, i.e., we have $\mu(ab) = \mu(a)\mu(b)$ whenever $a$ and $b$ are positive coprime integers.\\
{\footnotesize Hint: proceed by cases, taking cue from the definition of $\mu$.}
\item[(c)] Let $n > 1$ be any integer greater than $1$. Prove that
\[\sum_{d\in \mathscr{D}(n)} \mu(d) = 0.\]
In other words, the sum of $\mu(d)$, as $d$ ranges over all the positive divisors of $n$, is equal to $0$. 
\end{itemize}
\end{problem}

\vspace*{0.1in}

\begin{problem}\label{problem 4.8}
Let $f(n)$ and $g(n)$ be two complex-valued multiplicative functions of positive integers $n$. Define $h(n)$ by the formula:
\[h(n) = \sum_{d\in \mathscr{D}(n)}f(d)g\left(\frac{n}{d}\right).\]
\begin{itemize}
\item[(a)] Consider the case where $g(n) = \mu(n)$ defined in Problem \ref{problem 4.7} above and $f(n) = n$. Compute the values of $h(n)$ for $n = 1,\ldots,12$.
\item[(b)] Prove, in the general case of multiplicative functions $f$ and $g$, that $h$ is a multiplicative function.
\end{itemize}
(The new function $h$ is called the \textit{convolution} of $f$ and $g$, and is denoted by $f \star g$. The idea originates from Fourier analysis.)
\end{problem}